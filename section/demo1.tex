\newpage
\section{Demo: Fighter Jet (Deprecated)}
Our fighter jet has advanced communication capabilities and multiple redundant computers to enhance reliability and safety. 
To optimize space, weight, and power (SWaP), multiple components are integrated into a single electronic control unit (ECU).

\subsection{Components}
Our aircraft includes the following key components:
\begin{enumerate}[label=P\arabic*)]
	\item Flight Control System (FCS): 
		Interprets pilot inputs and controls the aircraft's flight surfaces, ensuring stable and responsive flight.
	\item Propulsion and Fuel System: 
		Manages the aircraft's engines, fuel tanks, pumps, and other related components to optimize performance and efficiency.
	\item Communication Systems: 
		Includes various types of radios for civil, domestic military, and international military communications, ensuring secure and reliable communication.
	\item Navigation Systems: 
		Utilizes Global Navigation Satellite System (GNSS) and Inertial Navigation System (INS) for precise positioning and navigation.
	\item Intelligence, Surveillance, and Reconnaissance (ISR): 
		Comprises radar, Signals Intelligence (SIGINT), Electronic Intelligence (ELINT), and Infrared (IR) systems for comprehensive situational awareness and threat detection.
	\item Weapons and Defensive Systems: 
		Includes offensive weapons and defensive countermeasures to enhance mission effectiveness and survivability.
	\item Life Support System: 
		Ensures the safety and comfort of the crew by managing oxygen supply, cabin pressure, and other environmental controls.
	\item Power System: 
		Provides and manages electrical power to various parts of the aircraft, including generators and batteries, to support all onboard systems.
\end{enumerate}


\subsection{Component Timing Constraints}
The following timing constraints are critical for the optimal performance of our components:
\begin{enumerate}[label=T\arabic*)]
	\item Flight Control System (FCS): Must process pilot inputs and adjust flight surfaces within 10-50 ms.
	\item Propulsion and Fuel System: Adjust engine and fuel parameters within 50-200 ms.
	\item Communication Systems: Maximum 50-100 ms latency for message transmission; 100-500 ms latency for ground station communication.
	\item Navigation Systems: Provide position updates at intervals of 100-200 ms.
	\item ISR Systems: Complete radar scans within 1 second; process SIGINT and ELINT data within 50-100 ms.
	\item Weapons and Defensive Systems: Acquire and track targets within 1-2 seconds; deploy weapons and countermeasures within 100 ms of command or threat detection.
	\item Life Support System: Monitor and adjust oxygen supply and cabin pressure in real-time, with updates every 100-200 ms.
	\item Power System: Manage power distribution to critical systems in real-time, with adjustments made within 50-100 ms.
\end{enumerate}

\subsection{Component Criticality}
The criticality of each component is classified as follows:
\begin{enumerate}[label=C\arabic*)]
	\item Flight Control System (FCS): Hard real-time. Missing a deadline could lead to loss of control and potential crashes.
	\item Propulsion and Fuel System: Hard real-time. Missing a deadline can lead to engine failure or inefficient fuel distribution.
	\item Communication Systems: Hard real-time for sensor fusion; soft real-time for general communication.
	\item Navigation Systems: Firm real-time. Occasional delays may degrade positioning accuracy but do not cause system failure.
	\item ISR Systems: Soft and hard real-time. Delays in processing sensor data may degrade intelligence quality but do not compromise safety. However, delays may also miss threats and targets, depending on the system's role.
	\item Weapons and Defensive Systems: Hard real-time. Delays may result in missed opportunities or even loss of the aircraft.
	\item Life Support System: Hard real-time.
	\item Power System: Hard real-time.
\end{enumerate}

\subsection{Role of S3K}

\begin{itemize}
	\item Isolation and safety. 

		\begin{itemize}
			\item Partitioning: 
				The S3K kernel divides the system into isolated partitions, each running critical tasks independently. 
				This isolation ensures that critical functions, such as the Flight Control System and Life Support System, are protected from faults in other partitions.
			\item Fault Containment: 
				By isolating critical functions, the system can contain and manage faults more effectively, preventing cascading failures and ensuring that essential systems remain operational.
		\end{itemize}
	\item Deterministic Behavior. 
		\begin{itemize}			
			\item Precise Control: 
				S3K allows for precise control over when and where a partition executes, ensuring that hard real-time deadlines are met. 
				This deterministic behavior is crucial for systems like the Flight Control System, where missing a deadline could lead to loss of control.
			\item Consistent Behavior: 
				The precise control ensures that each partition runs deterministically at the correct time, enabling error analysis, reproducibility, and consistent performance. 
				This is essential for meeting timing constraints and ensuring reliable operation.
			\item Predictable microarchitecture: 
				By flushing the microarchitectural state using the temporal fence before dispatching a partition, S3K can fix the possible states of a partition. 
				This process improves the partition's predictability and can make its timing behavior more deterministic.
		\end{itemize}
	\item Dynamic Resource Management. 
		\begin{itemize}
			\item Resource Allocation: 
				The kernel can dynamically allocate CPU time and memory to partitions based on their criticality and current load, optimizing resource utilization. 
				Critical components can be given higher priority to ensure they always have the resources needed to meet their deadlines.
			\item Adaptive Scheduling: 
				The system can reallocate resources based on mission demands. For example:
				\begin{itemize}
					\item Dogfight Scenario: 
						During a dogfight, the system may reallocate execution time from the communication or navigation system to the Flight Control, Propulsion, and Fuel System to increase responsiveness.
					\item Take-off and Landing: 
						During take-off and landing, the system may prioritize communication, flight control, propulsion, and fuel systems over weapons and defensive systems.
					\item Reconnaissance Missions: 
						During reconnaissance, the system may allocate more execution time to ISR and navigation systems to improve intelligence data.
					\item Cruising Over Friendly Territory: 
						The system can redistribute execution time from navigation, communication, weapons, and defensive systems to the fuel system to improve efficiency.
				\end{itemize}
			\item Flexibility: 
				The dynamic nature of S3K allows the system to adapt to unforeseen circumstances, such as component damage, by reallocating resources or disconnecting affected components.
		\end{itemize}

	\item Modular design. 
		\begin{itemize}			
			\item Easy Integration: 
				S3K's modular design allows for easy integration of components and upgrades. 
				Each partition can be developed, tested, and maintained independently before being integrated into the system.
			\item On-the-Fly Adjustments: 
				The modular design also allows for the removal and addition of partitions in-flight, providing flexibility to adapt to changing mission requirements.
		\end{itemize}
	\item Timing Synchronization. 
		\begin{itemize}
			\item Multi-Core and Multi-CPU Synchronization: 
				S3K's schedules can be synchronized across multiple cores and CPUs, allowing for precise control over the execution environment and minimizing interference between partitions.
			\item Predictable Behavior: 
				This synchronization improves the predictability of partition performance and ensures that deadlines are met, even when multiple systems run in parallel. 
				For example, synchronization prevents the communication system from interfering with the Flight Control System, ensuring that critical deadlines are not missed.
			\item Synchronized Redundancy: 
				For safety-critical components, redundancy is essential. 
				The failure of the Flight Control System, for example, can lead to catastrophic failure. 
				To mitigate this risk, multiple Flight Control Systems are run in parallel, and their state and outputs are compared.
				S3K's synchronized schedule allows these systems to run in parallel and maintain consistent states, enhancing the reliability and safety of the overall system.
		\end{itemize}
	\item Criticality separation.
		\begin{itemize}			
			\item Mixed Scheduling: 
				S3K allows for the separation of tasks of different criticality into different partitions. 
				For instance, critical functions like Flight Control and Life Support can be statically scheduled to ensure they meet their hard deadlines.
				Less critical systems, such as communication and navigation, can be grouped into partitions using soft real-time schedulers, optimizing resource usage while ensuring that critical tasks are prioritized.
		\end{itemize}
	\item Information Security.
		\begin{itemize}
			\item Spatial Isolation: 
				S3K hinders information flow between components by preventing them from accessing resources they are not authorized to access.
			\item Time Protection:
				S3K's time protection hinders information from flowing from other components by intra-core side-channels.
				This ensures that sensitive data can only be accessed through known channels.
			\item Third-party Components:
				Spatial isolation and time protection ensures that we can integrate and isolate less trusted third-party components.
				If the third-party component is compromised, the spatial and temporal isolation provided by S3K ensures that the rest of the system is secure and operational.
				This allows for a greater flexibility in system design as we can chose more components to integrate into our system.
		\end{itemize}
\end{itemize}

