\newpage
\section{Demo: Autonomous Reconnaissance Drone}
Our autonomous reconnaissance drone features advanced communication capabilities and a specialized configuration with multiple redundant computers to enhance reliability and safety. 
To optimize space, weight, and power (SWaP), multiple components are integrated into a single electronic control unit (ECU).

\subsection{Components}
Our aircraft includes the following key components:
\begin{enumerate}[label=P\arabic*)]
	\item Flight Control System: 
		Interprets control inputs to adjust the aircraft's flight surfaces, engine, fuel and energy systems.
	\item Autonomous System: Processes sensor data and commands to make decisions on flight path and intelligence gathering, enabling autonomous operation.
	\item Communication Systems: For communicating with a ground station to receive general commands.
	\item Navigation and Positioning Systems: Utilizes Global Navigation Satellite System (GNSS) and Inertial Navigation System (INS) for precise positioning and navigation.
	\item Intelligence, Surveillance, and Reconnaissance (ISR) System: Signals Intelligence (SIGINT), Electronic Intelligence (ELINT), and Infrared (IR) systems for comprehensive intelligence gathering.
\end{enumerate}

In Table \ref{tbl:drone}, we have the timing constraints and criticalities of the components summerized.

\begin{figure}[ht]
	\centering
	\begin{tabular}{c | c | c | c}
		Component & Max Latency & WCET & Criticality \\\hline
		Flight Control System & 50 ms & 5 ms & Hard real-time \\
		Autonomous System & 200 ms & 50 ms & Hard real-time \\
		Communication System & 200 ms & 30 ms & Hard/Firm real-time \\
		Nav.\ \& Pos.\ System & 200 ms & 40 ms & Firm real-time \\
		ISR system & 100 ms & 20 ms & Soft real-time \\
	\end{tabular}
	\caption{Timing Constraints and Criticality of Components}
	\label{tbl:drone}
\end{figure}


\subsection{Role of S3K (To be revised)}
\begin{itemize}
	\item Isolation and safety. 

		\begin{itemize}
			\item Partitioning: 
				The S3K kernel divides the system into isolated partitions, each running critical tasks independently. 
				This isolation ensures that critical functions, such as the Flight Control System, are protected from faults in other partitions.
			\item Fault Containment: 
				By isolating critical functions, the system can contain and manage faults more effectively, preventing cascading failures and ensuring that essential systems remain operational.
		\end{itemize}
	\item Deterministic Behavior. 
		\begin{itemize}			
			\item Precise Control: 
				S3K allows for precise control over when and where a partition executes, ensuring that hard real-time deadlines are met. 
				This deterministic behavior is crucial for systems like the Flight Control System, where missing a deadline could lead to loss of control.
			\item Consistent Behavior: 
				The precise control ensures that each partition runs deterministically at the correct time, enabling error analysis, reproducibility, and consistent performance. 
				This is essential for meeting timing constraints and ensuring reliable operation.
			\item Predictable microarchitecture: 
				By flushing the microarchitectural state using the temporal fence before dispatching a partition, S3K can fix the possible states of a partition. 
				This process improves the partition's predictability and can make its timing behavior more deterministic.
		\end{itemize}
	\item Dynamic Resource Management. 
		\begin{itemize}
			\item Resource Allocation: 
				The kernel can dynamically allocate CPU time and memory to partitions based on their criticality and current load, optimizing resource utilization. 
				Critical components can be given higher priority to ensure they always have the resources needed to meet their deadlines.
			\item Adaptive Scheduling: 
				The system can reallocate resources based on mission demands. For example:
				\begin{itemize}
					\item Take-off and Landing: 
						During take-off and landing, the system may prioritize communication, flight control, propulsion, and fuel systems over ISR.
					\item Reconnaissance Missions: 
						During reconnaissance, the system may allocate more execution time to ISR and navigation systems to improve intelligence gathering.
				\end{itemize}
			\item Flexibility: 
				The dynamic nature of S3K allows the system to adapt to unforeseen circumstances, such as component damage, by reallocating resources or disconnecting affected components.
		\end{itemize}

	\item Modular design. 
		\begin{itemize}			
			\item Easy Integration: 
				S3K's modular design allows for easy integration of components and upgrades. 
				Each partition can be developed, tested, and maintained independently before being integrated into the system.
			\item On-the-Fly Adjustments: 
				The modular design also allows for the removal and addition of partitions in-flight, providing flexibility to adapt to changing mission requirements.
		\end{itemize}
	\item Timing Synchronization. 
		\begin{itemize}
			\item Multi-Core and Multi-CPU Synchronization: 
				S3K's schedules can be synchronized across multiple cores and CPUs, allowing for precise control over the execution environment and minimizing interference between partitions.
			\item Predictable Behavior: 
				This synchronization improves the predictability of partition performance and ensures that deadlines are met, even when multiple systems run in parallel. 
				For example, synchronization prevents the communication system from interfering with the Flight Control System, ensuring that critical deadlines are not missed.
			\item Synchronized Redundancy: 
				For safety-critical components, redundancy is essential. 
				The failure of the Flight Control System, for example, can lead to catastrophic failure. 
				To mitigate this risk, multiple Flight Control Systems are run in parallel, and their state and outputs are compared.
				S3K's synchronized schedule allows these systems to run in parallel and maintain consistent states, enhancing the reliability and safety of the overall system.
		\end{itemize}
	\item Criticality separation.
		\begin{itemize}			
			\item Mixed Scheduling: 
				S3K allows for the separation of tasks of different criticality into different partitions. 
				For instance, critical functions like Flight Control can be statically scheduled to ensure they meet their hard deadlines.
				Less critical systems, such as communication and navigation, can be grouped into partitions using soft or firm real-time schedulers, optimizing resource usage while ensuring that critical tasks are prioritized.
		\end{itemize}
	\item Information Security.
		\begin{itemize}
			\item Spatial Isolation: 
				S3K hinders information flow between components by preventing them from accessing resources they are not authorized to access.
			\item Time Protection:
				S3K's time protection hinders information from flowing from other components by intra-core side-channels.
				This ensures that sensitive data can only be accessed through known channels.
			\item Third-party Components:
				Spatial isolation and time protection ensures that we can integrate and isolate less trusted third-party components.
				If the third-party component is compromised, the spatial and temporal isolation provided by S3K ensures that the rest of the system is secure and operational.
				This allows for a greater flexibility in system design as we can chose more components to integrate into our system.
		\end{itemize}
\end{itemize}

\subsection{Demonstrator Implementation}

\subsubsection{Components}

We implement an abstracted version of the reconnaissance drone in S3K using five components with varying criticalities. 
The system features a system monitor that oversees the overall allocation of resources to these components. 
Each component, in turn, has its own monitor that manages resource distribution to its subcomponents.

This hierarchical resource delegation enables a modular approach to system design and implementation. 
The system monitor allocates sufficient resources to component monitors, which then handle more detailed resource management. 
This structure allows for the addition, removal, and toggling of components and resources via the system monitor implemented to follow a simple policy regarding minimal resource constraints. 
The specifics of resource management are handled at the component level.

This modular, approach allows us to integrate components seamlessly. 
The system monitor only needs to know their minimum resource requirements and the scheduling constraints, enabling efficient and secure integration.

We have the following components:
\begin{itemize}
	\item Autonomy Component ($A$): Checks the system state, mission parameters and more. Sends directives to relevant components.
	\item Flight Control Component ($F$): Flight Control, Propulsion, Fuel, and Power systems.
	\item Navigation and Positioning Component ($N$): Navigation system using GNSS, INS, or CNS. 
	\item Communication Component ($C$): Communication system using satallite, radio, WiFi, and more.
	\item ISR Component ($R$): A third-party ISR system.
\end{itemize}
\todo{Object detection component?}

\subsubsection{Scheduling Specifications}
\begin{description}
	\item[System monitor.] Hard real-time scheduling.
		\begin{itemize}
			\item Always scheduled to receive directives from autonomous, communication or flight control system. Distributes timing resources according to directives.
			\item Implements a policy that constrains how resources may be distributed.
			\item When the monitor changes the schedule of a component, or enables a component, then the monitor yields to the component's monitor.
		\end{itemize}
	\item[Autonomy component.] Toggled hard real-time scheduling.
		\begin{itemize}
			\item System monitor may disable this component if the drone is remotely controlled.
			\item If remote connection is lost, the system monitor schedules the autonomy component.
			\item When enabled, the autonomy component require a minimum amount of execution time to meet its deadlines. More execution time can be provided to improve performance.
		\end{itemize}
	\item[Flight Control Component.] Toggled hard real-time scheduling.
		\begin{itemize}
			\item Always scheduled statically when in-flight.
			\item System monitor may disable this component when the drone is on the ground.
			\item May send directive to system monitor to indicate an emergency situation (loss of power, fuel, malfunction), requiring a different scheduling.
		\end{itemize}
	\item[Communication Component.] Toggled firm or hard real-time scheduling.
		\begin{itemize}
			\item This component may be disabled when communications are being jammed, or when the drone goes into a stealth mode.
			\item When this component is disabled, the autonomy component must be enabled.
		\end{itemize}
	\item[Navigation Component.] Firm real-time scheduling.
		\begin{itemize}
			\item In a stealth mode, the component may use inertial and celestial navigation.
			\item System monitor may add or remove execution time from this component. When this occurs, the system monitor notifies the navigation monitor.
		\end{itemize}
	\item [ISR Component.] Soft real-time component. Best-effort scheduling.
		\begin{itemize}
			\item A third-party component. We do not know the exact details of how it works.
		\end{itemize}
\end{description}

\begin{figure}
	\centering
	\begin{tabular}{c | c }
		Slots & Component \\\hline
		0-20 & System Monitor (1) \\
		20-40 & ISR (2) \\
		40-45 & FCS (1) \\
		45-85 & Autonomous (1) \\
		90-95 & FCS (2) \\
		95-100 & Autonomous (1) \\
		100-120 & Comm (1) \\
		120-140 & ISR (2) \\
		140-145 & FCS (3) \\
		145-155 & Comm (1) \\
		155-190 & Nav (1) \\
		190-195 & FCS (4) \\
		195-200 & Nav (1) \\
	\end{tabular}
	\caption{Scheduling of Drone. A major frame consists of 200 slots. Each slot represents 1 ms. One major frame represents 200 ms of execution time.}
\end{figure}
